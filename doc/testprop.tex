\Chapter{Testing properties of quasigroups and loops}

The reader should be aware that although loops are quasigroups, it is often the
case in the literature that a property named $P$ can differ for quasigroups and
loops. For instance, a Steiner loop is not necessarily a Steiner quasigroup.

To avoid such ambivalences, we often include the noun `Loop' or `Quasigroup' as
part of the name of the property, e.g. `IsSteinerQuasigroup' versus
`IsSteinerLoop'.

On the other hand, some properties coincide for quasigroups and loops and we
therefore do not include `Loop', `Quasigroup' as part of the name of the
property, e.g. `IsCommutative'.

%%%%%%%%%%%%%%%%%%%%%%%%%%%%%%%%%%%%%%%%%%%%%%%%%%%%%%%%%%%%%%%%%%%%%%%%%%%%%%%
\Section{Associativity, commutativity and generalizations}

The following properties test if a quasigroup $Q$ is associative and
commutative:

\>IsAssociative( <Q> ) P
\>IsCommutative( <Q> ) P

A loop $L$ is said to be <power-associative>\index{power-associative loop}
(resp.\ <diassociative>\index{diassociative loop}) if every monogenic subloop
of $L$ (resp. every $2$-generated subloop of $L$) is a group.

\>IsPowerAssociative( <L> ) P
\>IsDiassociative( <L> ) P

%%%%%%%%%%%%%%%%%%%%%%%%%%%%%%%%%%%%%%%%%%%%%%%%%%%%%%%%%%%%%%%%%%%%%%%%%%%%%%%
\Section{Inverse properties}

A loop $L$ has the <left inverse property>
%
\index{left inverse property}
%
 if $x^\lambda(xy)=y$
for every $x$, $y\in L$, where $x^\lambda$ is the left inverse of $x$. Dually,
$L$ has the <right inverse property>
%
\index{right inverse property}
%
 if $(yx)x^\rho=y$ for every $x$,
$y\in L$, where $x^\rho$ is the right inverse of $x$. If $L$ has both the left
and right inverse properties, it has the <inverse property>
%
\index{inverse property}
%
. We say that $L$
has <two-sided inverses>
%
\index{two-sided inverses}
%
 if $x^\lambda=x^\rho$ for every $x\in L$.

\>HasLeftInverseProperty( <L> ) P
\>HasRightInverseProperty( <L> ) P
\>HasInverseProperty( <L> ) P
\>HasTwosidedInverses( <L> ) P

A loop has the <weak inverse property>
%
\index{weak inverse property}
%
 if $(xy)^\lambda x = y^\lambda$.
Equivalently, a loop has the weak inverse property if $x(yx)^\rho = y^\rho$.

\>HasWeakInverseProperty( <L> ) P

According to \cite{Ar}, a loop $L$ has the <automorphic inverse property>
%
\index{automorphic inverse property}
%

if $(xy)^\lambda = x^\lambda y^\lambda$, or, equivalently, $(xy)^\rho = x^\rho
y^\rho$. (In particular, when $L$ has two-sided inverses and the automorphic
inverse property, it satisfies $(xy)^{-1}=x^{-1}y^{-1}$.) Similarly, $L$ has
the <antiautomorphic inverse property>
%
\index{antiautomorphic inverse property}
%
 if $(xy)^\lambda=y^\lambda
x^\lambda$, or, equivalently, $(xy)^\rho = y^\rho x^\rho$.

\>HasAutomorphicInverseProperty( <L> ) P
\>HasAntiautomorphicInverseProperty( <L> ) P

The following implications among inverse properties hold and are
implemented in {\LOOPS}:
\beginlist%unordered
\item{$\circ$}
    Inverse property implies left and right inverse properties,
    two-sided inverses, weak inverse property, and antiautomorphic
    inverse property.
\item{$\circ$}
    Antiautomorphic inverse property loops have two-sided inverses.
\item{$\circ$}
    If a loop has any two of the left inverse property, right inverse property,
    weak inverse property or antiautomorphic inverse property, it also has
    the inverse property.
\endlist

%%%%%%%%%%%%%%%%%%%%%%%%%%%%%%%%%%%%%%%%%%%%%%%%%%%%%%%%%%%%%%%%%%%%%%%%%%%%%%%
\Section{Some properties of quasigroups}

A quasigroup $Q$ is <semisymmetric>\index{semisymmetric quasigroup} if
$(xy)x=y$ for every $x$, $y\in Q$. Equivalently, $Q$ is semisymmetric if
$x(yx)=y$ for every $x$, $y\in Q$. A semisymmetric commutative quasigroup is
known as <totally symmetric>\index{totally symmetric quasigroup}. Totally
symmetric quasigroups are precisely quasigroups satisfying $xy=x\backslash y =
x/y$.

\>IsSemisymmetric( <Q> ) P
\>IsTotallySymmetric( <Q> ) P

A quasigroup $Q$ is <idempotent>\index{idempotent quasigroup} if $x^2=x$ for
every $x\in Q$. Idempotent totally symmetric quasigroups are known as <Steiner
quasigroups>\index{Steiner quasigroup}. A quasigroup $Q$ is
<unipotent>\index{unipotent quasigroup} if $x^2=y^2$ for every $x$, $y\in Q$.

\>IsIdempotent( <Q> ) P
\>IsSteinerQuasigroup( <Q> ) P
\>IsUnipotent( <Q> ) P

A quasigroup is <left distributive>\index{left distributive quasigroup} if it
satisfies $x(yz)=(xy)(xz)$. Similarly, it is <right distributive>\index{right
distributive quasigroup} if it satisfies $(xy)z=(xz)(yz)$. A <distributive quasigroup>
%
\index{distributive quasigroup}
%
 is a quasigroup that is both left and right distributive. A
quasigroup is called <entropic>\index{entropic quasigroup} or
<medial>\index{medial quasigroup} if it satisfies $(xy)(zw)=(xz)(yw)$.

\>IsLeftDistributive( <Q> ) P
\>IsRightDistributive( <Q> ) P
\>IsDistributive( <Q> ) P
\>IsEntropic( <Q> ) P
\>IsMedial( <Q> ) P

In order to be compatible with {\GAP}'s terminology, we also support the synonyms

\>IsLDistributive( <Q> ) P
\>IsRDistributive( <Q> ) P

for `IsLeftDistributive' and `IsRightDistributive'. respectively.

%%%%%%%%%%%%%%%%%%%%%%%%%%%%%%%%%%%%%%%%%%%%%%%%%%%%%%%%%%%%%%%%%%%%%%%%%%%%%%%
\Section{Loops of Bol-Moufang type and related properties}

Following \cite{Fe} and \cite{PV}, a variety of loops is said to be of
<Bol-Moufang type>\index{loops of Bol-Moufang type} if it is defined by a
single <identity of Bol-Moufang type>
%
\index{identity of Bol-Moufang type}
%
, i.e., by an identity that:
\beginlist%unordered
\item{$\circ$}
    contains the same $3$ variables on both sides,
\item{$\circ$}
     exactly one of the variables occurs twice on both sides,
\item{$\circ$}
     the variables occur in the same order on both sides.
\endlist
It is proved in \cite{PV} that there are $13$ varieties of nonassociative loops
of Bol-Moufang type. These are:
\beginlist%unordered
\item{$\circ$}
    <left alternative loops>
%
\index{left alternative loops}
%
, defined by $x(xy) = (xx)y$,
\item{$\circ$}
    <right alternative loops>
%
\index{right alternative loops}
%
, defined by $x(yy) = (xy)y$,
\item{$\circ$}
    <left nuclear square loops>
%
\index{left nuclear square loops}
%
, defined by $(xx)(yz) = ((xx)y)z$,
\item{$\circ$}
    <middle nuclear square loops>
%
\index{middle nuclear square loops}
%
, defined by $x((yy)z) = (x(yy))z$,
\item{$\circ$}
    <right nuclear square loops>
%
\index{right nuclear square loops}
%
, defined by $x(y(zz)) = (xy)(zz)$,
\item{$\circ$}
    <flexible loops>
%
\index{flexible loops}
%
, defined by $x(yx) = (xy)x$,
\item{$\circ$}
    <left Bol loops>
%
\index{left Bol loops}
%
, defined by $x(y(xz)) = (x(yz))x$,
    always left alternative,
\item{$\circ$}
    <right Bol loops>
%
\index{right Bol loops}
%
, defined by $x((yz)y) = ((xy)z)y$,
    always right alternative,
\item{$\circ$}
    <LC-loops>
%
\index{LC-loops}
%
, defined by $(xx)(yz) = (x(xy))z$,
    always left alternative, left and middle nuclear square,
\item{$\circ$}
    <RC-loops>
%
\index{RC-loops}
%
, defined by $x((yz)z) = (xy)(zz)$,
    always right alternative, right and middle nuclear square,
\item{$\circ$}
    <Moufang loops>
%
\index{Moufang loops}
%
, defined by $(xy)(zx) = (x(yz))x$,
    always flexible, left and right Bol,
\item{$\circ$}
    <C-loops>
%
\index{C-loops}
%
, defined by $x(y(yz)) = ((xy)y)z$,
    always LC and RC,
\item{$\circ$}
    <extra loops>
%
\index{extra loops}
%
, defined by $x(y(zx)) = ((xy)z)x$,
    always Moufang and C.
\endlist
Note that although some of the defining identities are not of Bol-Moufang type,
they are equivalent to a Bol-Moufang identity. Moreover, many varieties are
defined in several ways, by equivalent identities of Bol-Moufang type.

There are several varieties related to loops of Bol-Moufang type. A loop is
said to be <alternative>\index{alternative loop} if it is both left and
right alternative, and <nuclear square>\index{nuclear square loop} if it is left,
middle and right nuclear square.

Here are the corresponding {\LOOPS} commands (argument $L$ indicates that the
property applies only to loops, argument $Q$ indicates that the property
applies also to quasigroups):

\>IsExtraLoop( <L> ) P
\>IsMoufangLoop( <L> ) P
\>IsCLoop( <L> ) P
\>IsLeftBolLoop( <L> ) P
\>IsRightBolLoop( <L> ) P
\>IsLCLoop( <L> ) P
\>IsRCLoop( <L> ) P
\>IsLeftNuclearSquareLoop( <L> ) P
\>IsMiddleNuclearSquareLoop( <L> ) P
\>IsRightNuclearSquareLoop( <L> ) P
\>IsNuclearSquareLoop( <L> ) P
\>IsFlexible( <Q> ) P
\>IsLeftAlternative( <Q> ) P
\>IsRightAlternative( <Q> ) P
\>IsAlternative( <Q> ) P

While listing the varieties of loops of Bol-Moufang type, we have also listed
all inclusions among them. These inclusions are built into {\LOOPS}.

The following trivial example shows some of the implications and the naming
conventions of {\LOOPS} at work:

\beginexample
gap> L := LoopByCayleyTable( [ [ 1, 2 ], [ 2, 1 ] ] );
<loop of order 2>
gap> [ IsLeftBolLoop( L ), L ]
[ true, <left Bol loop of order 2> ]
gap> [ HasIsLeftAlternativeLoop( L ), IsLeftAlternativeLoop( L ) ];
[ true, true ]
gap> [ HasIsRightBolLoop( L ), IsRightBolLoop( L ) ];
[ false, true ]
gap> L;
<Moufang loop of order 2>
gap> [ IsAssociative( L ), L ];
[ true, <associative loop of order 2> ]
\endexample

The analogous terminology for quasigroups of Bol-Moufang type is not
standard yet, and hence is not supported in {\LOOPS}.


%%%%%%%%%%%%%%%%%%%%%%%%%%%%%%%%%%%%%%%%%%%%%%%%%%%%%%%%%%%%%%%%%%%%%%%%%%%%%%%
\Section{Conjugacy closed loops and related properties}

\noindent A loop is <left> (resp.\ <right>) <conjugacy
closed>\index{left conjugacy closed loop}\index{right conjugacy closed loop} if
its left (resp.\ right) translations are closed under composition. A loop that
is both left and right conjugacy closed is called <conjugacy
closed>\index{conjugacy closed loop}. It is common to refer to these loops as
LCC-, RCC-, CC-loops, respectively.

\>IsLCCLoop( <L> ) P
\>IsRCCLoop( <L> ) P
\>IsCCLoop( <L> ) P

The equivalence LCC $+$ RCC $=$ CC is built into {\LOOPS}.

A loop is <Osborn>\index{Osborn loop} if it satisfies $x(yz\cdot
x)=(x^\lambda\backslash y)(zx)$, where $x^\lambda$ is the left inverse of $x$. Both
Moufang loops and CC-loops are Osborn.

\>IsOsbornLoop( <L> ) P

%%%%%%%%%%%%%%%%%%%%%%%%%%%%%%%%%%%%%%%%%%%%%%%%%%%%%%%%%%%%%%%%%%%%%%%%%%%%%%%
\Section{Additional varieties of loops}

A left (resp. right) Bol loop with the automorphic inverse property
is known as <left> (resp. <right>) <Bruck loop>\index{left Bruck
loop}\index{right Bruck loop}. Bruck loops are also known as
<K-loops>\index{K-loop}.

\>IsLeftBruckLoop( <L> ) P
\>IsLeftKLoop( <L> ) P
\>IsRightBruckLoop( <L> ) P
\>IsRightKLoop( <L> ) P

<Steiner loop>
%
\index{Steiner loop}
%
 is an inverse property loop of exponent $2$.

\>IsSteinerLoop( <L> ) P
